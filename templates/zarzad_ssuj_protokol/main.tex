\documentclass[10pt, a4paper]{article}
\usepackage[explicit]{titlesec}
\usepackage[margin=1in]{geometry}
\usepackage{fontspec}
\usepackage{polski}
\usepackage{polyglossia}
\setdefaultlanguage{polish}
\usepackage{graphicx}
\defaultfontfeatures{Ligatures=TeX}
\setmainfont{TeX Gyre Termes}
\usepackage{chngcntr}
\usepackage{enumitem}
\usepackage{calc}
\usepackage[final]{pdfpages}
\usepackage{parskip}

\titleformat{\paragraph}{\centering\normalfont\large\bfseries}{}{0em}{#1 \theparagraph}
\newcommand{\lexarticle}{\paragraph{\S}}
\counterwithout{paragraph}{subsubsection}
\setcounter{secnumdepth}{4}

\newcommand{\VAR}[1]{\{\{ #1 \}\}}
\newcommand{\BLOCK}[1]{\{\% #1 \$\}}

\title{Protokół}

\begin{document}
    \begin{center}
        \includegraphics[height=4cm]{SS_UJ_CMYK_symetryczny}
    \end{center}

    \begin{center}
        \large
        \textbf{Protokół}\\
        \smallskip
        z posiedzenia Zarządu\\
        Samorządu Studentów Uniwersytetu Jagiellońskiego\\
        z dnia \VAR{date.strftime('%d %B %Y')} r. \\
        \normalsize
    \end{center}


    \section{Otwarcie posiedzenia i stwierdzenie kworum}
    O godzinie \VAR{startTime} posiedzenie otworzyła Przewodnicząca Samorządu Studentów UJ Katarzyna Jedlińska. Na podstawie listy obecności stwierdzono kworum.


    \section{Zatwierdzenie porządku obrad}
    Przedstawiono następujący porządek obrad.
    \begin{enumerate}
        \item Otwarcie posiedzenia i stwierdzenie kworum.
        \item Zatwierdzenie porządku obrad.
        \BLOCK{for project in projects}
        \item Przyznanie dotacji na projekt ,,\VAR{project.title}'' (\VAR{project.signature})
        \BLOCK{endfor}
        \item Wolne wnioski.
        \item Zamknięcie posiedzenia.
    \end{enumerate}

    Nie zostały zgłoszone uwagi co do porządku obrad.

    Porządek obrad został przyjęty jednogłośnie w głosowaniu jawnym.


    \BLOCK{for project in projects}


    \section{Przyznanie dotacji na projekt ,,\VAR{project.title}'' (\VAR{project.signature})}

    Sekretarz Zarządu przedstawiła wniosek o przyznanie \VAR{project.entity} dotacji ze środków przeznaczonych przez uczelnię na sprawy studenckie w wysokości \VAR{amount_spelled(project.amountRequested)} na realizację projektu o nazwie \textit{,,\VAR{project.title}''}, sygnatura \textbf{\VAR{project.signature}}.

    \VAR{project.remarks}

    Po dyskusji Zarząd zdecydował o poddaniu pod głosowanie jawne udzielenia dotacji w kwocie \VAR{amount_spelled(project.amountReceived)}.

    \begin{center}
        \textbf{Wyniki głosowania}\smallskip\\
        Oddanych głosów: \VAR{project.votesFor + project.votesAgainst + project.votesAbstain}\\
        Głosów za: \VAR{project.votesFor}\\
        Głosów przeciw: \VAR{project.votesAgainst}\\
        Głosów wstrzymujące: \VAR{project.votesAbstain}
    \end{center}

    Wniosek został \textbf{\BLOCK{if project.idx > 0}przyjęty\BLOCK{else}odrzucony\BLOCK{endif}}.

    \BLOCK{endfor}


    \section{Wolne wnioski}

    Nie zgłoszono wniosków.


    \section{Zamknięcie posiedzenia}

    Posiedzenie Zarządu Samorządu Studentów UJ zostało zamknięte o godzinie \VAR{endTime}.

    \bigskip \bigskip
    \hfill\begin{minipage}{\dimexpr\textwidth-11cm}
    \begin{flushright}
    \textbf{
    Protokołował\\
    Adam Pardyl}
    \end{flushright}
    \end{minipage}

    \vfill

    \footnotesize
    Załączniki:
    \begin{enumerate}
        \item Lista obecności.
        \BLOCK{for project in projects}
            \BLOCK{if project.idx > 0}
                \item Uchwała nr \VAR{project.idx} Zarządu Samorządu Studentów Uniwersytetu Jagiellońskiego z dnia \VAR{date.strftime('%d %B %Y')} r. w sprawie przyznania dotacji ze środków przeznaczonych przez uczelnię na sprawy studenckie.
            \BLOCK{endif}
        \BLOCK{endfor}
    \end{enumerate}
    \normalsize
\end{document}
