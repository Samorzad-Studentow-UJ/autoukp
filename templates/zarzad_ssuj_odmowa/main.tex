\documentclass[10pt, a4paper]{article}
\usepackage[explicit]{titlesec}
\usepackage[margin=1in]{geometry}
\usepackage{fontspec}
\usepackage{polski}
\usepackage{polyglossia}
\setdefaultlanguage{polish}
\usepackage{graphicx}
\defaultfontfeatures{Ligatures=TeX}
\setmainfont{TeX Gyre Termes}
\usepackage{chngcntr}
\usepackage{enumitem}
\usepackage{calc}
\usepackage[final]{pdfpages}

\titleformat{\paragraph}{\centering\normalfont\large\bfseries}{}{0em}{#1 \theparagraph.}
\newcommand{\lexarticle}{\paragraph{\S}}
\counterwithout{paragraph}{subsubsection}
\setcounter{secnumdepth}{4}

\newcommand{\VAR}[1]{\{\{ #1 \}\}}
\newcommand{\BLOCK}[1]{\{\% #1 \$\}}

\title{Uchwała}

\begin{document}
    \begin{center}
        \includegraphics[height=4cm]{SS_UJ_CMYK_symetryczny}
    \end{center}

    \begin{center}
        \large
        \textbf{Uchwała nr \VAR{project.idx}}\\
        \smallskip
        Zarządu Samorządu Studentów\\
        Uniwersytetu Jagiellońskiego\\
        z dnia \VAR{date.strftime('%d %B %Y')} r. \\
        \normalsize
        \textbf{
            \begin{description}[leftmargin=\widthof{w sprawie: .}]
                \item[w sprawie:]
                odmowy przyznania dotacji ze środków przeznaczonych przez uczelnię na sprawy studenckie.
            \end{description}}
    \end{center}

    Na podstawie \S 38 ust. 1 pkt 2 \textit{Regulaminu Samorządu Studentów Uniwersytetu Jagiellońskiego z dnia 26 października 2010 r. (z~późn. zm.)} uchwala się, co następuje:

    \lexarticle
    \textbf{Odmawia się} przyznania \VAR{project.entity} dotację ze środków przeznaczonych przez uczelnię na sprawy studenckie w wysokości \VAR{amount_spelled(project.amountReceived)} na realizację projektu o nazwie \textbf{,,\VAR{project.title}''}, o którym mowa we wniosku o sygnaturze \textbf{\mbox{\VAR{project.signature}}}.

    \lexarticle
    Uchwała wchodzi w życie z dniem podjęcia.

\end{document}
